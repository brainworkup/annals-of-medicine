% Options for packages loaded elsewhere
% Options for packages loaded elsewhere
\PassOptionsToPackage{unicode}{hyperref}
\PassOptionsToPackage{hyphens}{url}
\PassOptionsToPackage{dvipsnames,svgnames,x11names}{xcolor}
%
\documentclass[
  12pt,
  letterpaper,
]{article}
\usepackage{xcolor}
\usepackage[top=1in,bottom=1in,left=1in,right=1in]{geometry}
\usepackage{amsmath,amssymb}
\setcounter{secnumdepth}{-\maxdimen} % remove section numbering
\usepackage{iftex}
\ifPDFTeX
  \usepackage[T1]{fontenc}
  \usepackage[utf8]{inputenc}
  \usepackage{textcomp} % provide euro and other symbols
\else % if luatex or xetex
  \usepackage{unicode-math} % this also loads fontspec
  \defaultfontfeatures{Scale=MatchLowercase}
  \defaultfontfeatures[\rmfamily]{Ligatures=TeX,Scale=1}
\fi
\usepackage{lmodern}
\ifPDFTeX\else
  % xetex/luatex font selection
  \setmainfont[Numbers=Proportional,Numbers=OldStyle]{Linux Libertine O}
  \setmathfont[]{Libertinus Math}
\fi
% Use upquote if available, for straight quotes in verbatim environments
\IfFileExists{upquote.sty}{\usepackage{upquote}}{}
\IfFileExists{microtype.sty}{% use microtype if available
  \usepackage[]{microtype}
  \UseMicrotypeSet[protrusion]{basicmath} % disable protrusion for tt fonts
}{}
\usepackage{setspace}


\usepackage{longtable,booktabs,array}
\usepackage{calc} % for calculating minipage widths
% Correct order of tables after \paragraph or \subparagraph
\usepackage{etoolbox}
\makeatletter
\patchcmd\longtable{\par}{\if@noskipsec\mbox{}\fi\par}{}{}
\makeatother
% Allow footnotes in longtable head/foot
\IfFileExists{footnotehyper.sty}{\usepackage{footnotehyper}}{\usepackage{footnote}}
\makesavenoteenv{longtable}
\usepackage{graphicx}
\makeatletter
\newsavebox\pandoc@box
\newcommand*\pandocbounded[1]{% scales image to fit in text height/width
  \sbox\pandoc@box{#1}%
  \Gscale@div\@tempa{\textheight}{\dimexpr\ht\pandoc@box+\dp\pandoc@box\relax}%
  \Gscale@div\@tempb{\linewidth}{\wd\pandoc@box}%
  \ifdim\@tempb\p@<\@tempa\p@\let\@tempa\@tempb\fi% select the smaller of both
  \ifdim\@tempa\p@<\p@\scalebox{\@tempa}{\usebox\pandoc@box}%
  \else\usebox{\pandoc@box}%
  \fi%
}
% Set default figure placement to htbp
\def\fps@figure{htbp}
\makeatother





\setlength{\emergencystretch}{3em} % prevent overfull lines

\providecommand{\tightlist}{%
  \setlength{\itemsep}{0pt}\setlength{\parskip}{0pt}}





% -----------------------
% CUSTOM PREAMBLE STUFF
% -----------------------

% -----------------
% Typography tweaks
% -----------------
% Indent size
\setlength{\parindent}{0.5in}
\setlength{\leftmargin}{0.5in}

% Fix widows and orphans
\usepackage[all,defaultlines=2]{nowidow}

% List things
\usepackage{enumitem}
% Same document-level indentation for ordered and ordered lists
\setlist[1]{labelindent=\parindent}
\setlist[itemize]{leftmargin=*}
\setlist[enumerate]{leftmargin=*}

% Wrap definition list terms
% https://tex.stackexchange.com/a/9763/11851
\setlist[description]{style=unboxed}


% For better TOCs
\usepackage[titles]{tocloft}

% Remove left margin in lists inside longtables
% https://tex.stackexchange.com/a/378190/11851
\AtBeginEnvironment{longtable}{\setlist[itemize]{nosep, wide=0pt, leftmargin=*, before=\vspace*{-\baselineskip}, after=\vspace*{-\baselineskip}}}

% For fancy ORCID links
\usepackage{orcidlink}

% Indent all first paragraphs because APA wants that
\usepackage{indentfirst}

% Allow for /singlespacing and /doublespacing
\usepackage{setspace}


% -----------------
% Title block stuff
% -----------------

% Abstract
\usepackage{abstract}
\renewcommand{\abstractnamefont}{\normalfont\normalsize\bfseries}
\renewcommand{\abstracttextfont}{\normalsize}
\setlength{\absleftindent}{\parindent}
\setlength{\absrightindent}{\parindent}


% Keywords
\providecommand{\keywords}[1]{\textbf{\textit{Keywords---}}#1}
  
% Title
\usepackage{titling}
\setlength{\droptitle}{2\baselineskip}
\pretitle{\begin{center}\normalfont\normalsize\bfseries}
\posttitle{\par\end{center}\vspace{2\baselineskip}}


% ------------------
% Section headings
% ------------------
\usepackage{titlesec}
\titleformat*{\section}{\center\normalsize\bfseries}
\titleformat*{\subsection}{\normalsize\bfseries}
\titleformat*{\subsubsection}{\normalsize\bfseries\itshape}
\titleformat{\paragraph}[runin]{\bfseries}{\theparagraph.}{3pt}{\space}[.]
\titleformat{\subparagraph}[runin]{\bfseries\itshape}{\thesubparagraph.}{3pt}{\space}[.]

% \titlespacing{<command>}{<left>}{<before-sep>}{<after-sep>}
% Starred version removes indentation in following paragraph
\titlespacing*{\section}{0em}{2em}{0em}
\titlespacing*{\subsection}{0em}{1em}{0em}
\titlespacing*{\subsubsection}{0em}{0em}{0em}
\titlespacing*{\paragraph}{\parindent}{1ex}{1em}
\titlespacing*{\subparagraph}{\parindent}{1ex}{1em}


% -----------
% Footnotes
% -----------
% NB: footmisc has to come after setspace and biblatex because of conflicts
\usepackage[bottom]{footmisc}
\renewcommand*{\footnotelayout}{\footnotesize}

\addtolength{\skip\footins}{12pt}  % vertical space between rule and main text
\setlength{\footnotesep}{16pt}  % vertical space between footnotes


% ----------
% Captions
% ----------
\usepackage[font=normalsize]{caption}


% --------
% Macros
% --------
% pandoc will not convert text within \begin{} XXX \end{} to Markdown and will
% treat it as regular TeX. Because of this, it's impossible to do stuff like
% this:

% \begin{landscape}
%
% | One | Two   |
% |-----+-------|
% | my  | table |
% | is  | nice  |
%
% \end{landscape}
%
% Since it'll render like: | One | Two | |—–+——-| | my | table | | is | nice |
% 
% BUT, from this http://stackoverflow.com/a/41945462/120898 we can get around
% this by creating new commands for \begin and \end, like this:
\usepackage{pdflscape}
\newcommand{\blandscape}{\begin{landscape}}
\newcommand{\elandscape}{\end{landscape}}

% \blandscape
%
% | One | Two   |
% |-----+-------|
% | my  | table |
% | is  | nice  |
%
% \elandscape

% Same thing, but for generic groups
% But can't use \bgroup and \egroup because those are built-in TeX things
\newcommand{\stgroup}{\begingroup}
\newcommand{\fingroup}{\endgroup}


% ---------------------------
% END CUSTOM PREAMBLE STUFF
% ---------------------------
\makeatletter
\@ifpackageloaded{caption}{}{\usepackage{caption}}
\AtBeginDocument{%
\ifdefined\contentsname
  \renewcommand*\contentsname{Table of contents}
\else
  \newcommand\contentsname{Table of contents}
\fi
\ifdefined\listfigurename
  \renewcommand*\listfigurename{List of Figures}
\else
  \newcommand\listfigurename{List of Figures}
\fi
\ifdefined\listtablename
  \renewcommand*\listtablename{List of Tables}
\else
  \newcommand\listtablename{List of Tables}
\fi
\ifdefined\figurename
  \renewcommand*\figurename{Figure}
\else
  \newcommand\figurename{Figure}
\fi
\ifdefined\tablename
  \renewcommand*\tablename{Table}
\else
  \newcommand\tablename{Table}
\fi
}
\@ifpackageloaded{float}{}{\usepackage{float}}
\floatstyle{ruled}
\@ifundefined{c@chapter}{\newfloat{codelisting}{h}{lop}}{\newfloat{codelisting}{h}{lop}[chapter]}
\floatname{codelisting}{Listing}
\newcommand*\listoflistings{\listof{codelisting}{List of Listings}}
\makeatother
\makeatletter
\makeatother
\makeatletter
\@ifpackageloaded{caption}{}{\usepackage{caption}}
\@ifpackageloaded{subcaption}{}{\usepackage{subcaption}}
\makeatother
\usepackage{bookmark}
\IfFileExists{xurl.sty}{\usepackage{xurl}}{} % add URL line breaks if available
\urlstyle{same}
\hypersetup{
  pdftitle={Diagnosis of Attention-Deficit/Hyperactivity Disorder in Adults},
  pdfauthor={Julian Bashir; Elim Garak},
  pdfkeywords={surgery, espionage, brains},
  colorlinks=true,
  linkcolor={DarkSlateBlue},
  filecolor={Maroon},
  citecolor={DarkSlateBlue},
  urlcolor={DarkSlateBlue},
  pdfcreator={LaTeX via pandoc}}


% -----------------------
% END-OF-PREAMBLE STUFF
% -----------------------

% For patching commands like \subtitle
\usepackage{etoolbox}


% -----------------------
% Left aligned settings
% -----------------------
\usepackage[document]{ragged2e}
\setlength{\RaggedRightParindent}{\parindent}
\setlength{\RaggedRightRightskip}{0pt plus 3em}

% That also affects all the captions, so this forces them to be unindented
\AtBeginEnvironment{figure}{\setlength{\RaggedRightParindent}{0pt}}
\AtBeginEnvironment{table}{\setlength{\RaggedRightParindent}{0pt}}
\AtBeginEnvironment{longtable}{\setlength{\RaggedRightParindent}{0pt}}  % Longtables
\AtBeginEnvironment{apptbl}{\setlength{\RaggedRightParindent}{0pt}}  % My custom appendix section
\AtBeginEnvironment{tcolorbox}{\setlength{\RaggedRightParindent}{0pt}}  % Quarto callouts

% ----------
% Endnotes
% ----------
\usepackage{endnotes}
\renewcommand{\enotesize}{\normalsize}
\let\footnote=\endnote

% -----------------
% Running headers
% -----------------

\usepackage{fancyhdr}
\setlength{\headheight}{0.25in}
\renewcommand{\headrulewidth}{0pt}  % Remove lines
\renewcommand{\footrulewidth}{0pt}

% SHORT TITLE           Page #
\fancypagestyle{normal}{
  \fancyhf{}
  \lhead{\uppercase{ Cranial Implants }}
  \rhead{\thepage}
}

% Running head: SHORT TITLE        Page #
\fancypagestyle{title}{
  \fancyhf{}
  \lhead{Running head: \uppercase{ Cranial Implants }}
  \rhead{}
}

% Use regular heading style
\pagestyle{normal}


% ---------------------- 
% Title block elements
% ---------------------- 
\usepackage{authblk}
\renewcommand*{\Authsep}{, }
\renewcommand*{\Authand}{ and }
\renewcommand*{\Authands}{, and }
\renewcommand*{\Affilfont}{\normalsize}
\renewcommand*{\Authfont}{\normalsize}

\title{Diagnosis of Attention-Deficit/Hyperactivity Disorder in Adults:}

\makeatletter
\providecommand{\subtitle}[1]{% add subtitle to \maketitle
  \apptocmd{\@title}{\par #1 \par}{}{}
}
\makeatother
\subtitle{A Systematic Review}

\author[1]{Julian Bashir}
\author[1,2]{Elim Garak}

\affil[1]{Starbase Deep Space Nine}
\affil[2]{Terok Nor}

\date{}


% Typeset URLs in the same font as their parent environment
%
% This has to come at the end of the preamble, after any biblatex stuff because 
% some biblatex styles (like APA) define their own \urlstyle{}
\usepackage{url}
\urlstyle{same}


% Make sure math environments are always single spaced and have a little space after
\let\oldDisplayMath=\[
\let\endoldDisplayMath=\]
\renewcommand{\[}{\begin{singlespace}\oldDisplayMath}
\renewcommand{\]}{\endoldDisplayMath\end{singlespace}\vspace{\baselineskip}}

% ---------------------------
% END END-OF-PREAMBLE STUFF
% ---------------------------
\begin{document}
% ---------------
% TITLE SECTION
% ---------------

\begin{titlepage}
\center

% Don't include a newpage before \maketitle
{\let\newpage\relax\maketitle}

% Needs to come after \maketitle
\thispagestyle{title}

\vspace{0.25in}


\vspace{0.5in}

\begin{center}

\textbf{Author note}

\raggedright\setlength{\parindent}{0.5in}Julian
Bashir,~\orcidlink{0000-0002-3948-3914}~\url{https://orcid.org/0000-0002-3948-3914}, Chief
Medical Officer, Sick Bay, Starbase Deep Space Nine 

\raggedright\setlength{\parindent}{0.5in}Elim Garak, Shopkeeper and
Tailor, Sick Bay, Starbase Deep Space Nine; Promenade, Terok Nor 


\end{center}

\begin{center}

\textbf{Correspondence}

\par\raggedright\setlength{\parindent}{0.5in}Correspondence concerning
this article should be addressed to~Julian
Bashir (\href{mailto:jbashir@starfleet.ufp}{jbashir@starfleet.ufp}), Sick
Bay, Starbase Deep Space Nine, 1234 Main Street, Anytown, NY 90210, USA 

\end{center}

\begin{center}

\textbf{Additional information}

\par\raggedright\setlength{\parindent}{0.5in}We have no known conflict
of interest to disclose.

\end{center}

\begin{center}

\textbf{Acknowledgments}

\par\raggedright\setlength{\parindent}{0.5in}Placeholder text generated
at \url{https://vlad-saling.github.io/star-trek-ipsum/}.

\end{center}

\vfill  % Fill the rest of the page with whitespace

\end{titlepage}

\doublespacing

\begin{abstract}
\noindent Space: the final frontier. These are the voyages of the
starship \emph{Enterprise}. Its continuing mission: to explore strange
new worlds. To seek out new life and new civilizations. To boldly go
where no one has gone before!
\end{abstract}

\vspace{\baselineskip}

\indent \keywords{%
surgery, espionage, brains%
}

\newpage

\begin{center}
\singlespacing
\textbf{Diagnosis of Attention-Deficit/Hyperactivity Disorder in
Adults:\\A Systematic Review}
\end{center}

% -------------------
% END TITLE SECTION
% -------------------

\setstretch{2}
\subsection{Abstract}\label{abstract}

\textbf{Objectives.} This evidence report synthesizes the results of
evaluations of available tools for diagnosing attention
deficit/hyperactivity disorder in adults to inform patients, clinicians,
and policy makers.

\textbf{Review methods.} Following a detailed published protocol and
informed by a technical expert panel, we reviewed the evidence for
diagnostic tools. In October 2024, we searched nine research databases
from inception, research and guideline registries, reference-mined
existing reviews and practice guidelines, and consulted with experts to
identify evaluations that compared tools used for the diagnosis of ADHD
in people of 18 years or older to a clinical diagnosis. The review will
be updated during peer review. Registration CRD42025638106.

\textbf{Results.} We identified 117 studies evaluating the diagnostic
performance of self-report questionnaires, peer review questionnaires,
neuropsychological tests, neuroimaging, electroencephalogram (EEG),
diverse biomarkers, clinician tools, combinations of modalities, and
tools to identify feigning ADHD.

We found few direct performance comparisons between tests; the strength
of evidence (SoE) was often insufficient for evidence statements. There
was low SoE for lower clinical misdiagnosis rates (false positive rate
in clinical samples) for self-report versus both clinician tools and
neuropsychological tests, and for combinations of input versus
neuropsychological tests alone. For sensitivity, results favored
self-report and combinations of input over neuropsychological tests
alone and studies found no difference between self-reports and clinician
tools. For specificity, results favored combinations of input over
neuropsychological tests alone, and self-reports over clinician tools.

Combinations of input indicated a fair rate of clinical false positive
rates, good sensitivity, and acceptable specificity. Self-reports showed
good sensitivity and specificity, but often not both in the same study;
administration time was short, but agreement with other raters was
limited. Peer reports showed limited specificity. Neuropsychological
tests reported substantial false positive rates in clinical samples,
acceptable sensitivity and specificity, and short administration times.
The small number of neuroimaging studies and EEG studies reported
acceptable sensitivity and specificity, and short administration time.
Clinician tools reported fair sensitivity. All results were rated low
SoE. Results for all other key outcomes (e.g., diagnostic concordance
between primary care clinicians and specialists) were rated
insufficient, either due to lack of studies or wide variation in
results.

\textbf{Conclusions.} A substantial volume of research for diagnostic
performance of tests for ADHD in adults exists, in particular for
self-report questionnaires and neuropsychological tests. Multiple
different diagnostic modalities have been explored and combinations of
input appear particularly promising. Despite the volume, evidence was
insufficient for several key outcomes. Performance is associated with
the comparator and whether diagnostic tools aim to distinguish between
adults with ADHD and neurotypical adults, or adults with other clinical
conditions.

\begin{center}\rule{0.5\linewidth}{0.5pt}\end{center}

\subsection{1. Introduction}\label{introduction}

\subsubsection{1.1 Background}\label{background}

Attention-deficit/hyperactivity disorder (ADHD) is characterized by
persistent symptoms in the domains of inattention, hyperactivity, and
impulsivity that often begin in childhood. Clinically significant
symptoms, especially inattention, persist into adulthood in most
individuals. The lifetime prevalence of ADHD is approximately 5.3\%,
although epidemiological studies that have not required a childhood
onset have suggested that its prevalence in adults may be as high as
seven percent. Many adults with ADHD adopt lifestyles that help
compensate for their symptoms, they often need to exert excess energy to
overcome impairments. Impaired productivity because of poor time
management, procrastination, and distractibility can limit work
productivity and lower overall quality of life. Affected adults are
often distressed by their inability to realize their full potential and
by persistent symptoms of restlessness, erratic moods, and poor
self-esteem.

ADHD is most often first diagnosed in elementary or middle school age
years or, less commonly, in high school or college when increasing
academic demands surpass the attentional capacities of the affected
person. ADHD can also be first diagnosed in adulthood, when impairments
in attention, organization, and impulsivity produce recurrent problems
with occupational, social, or family functioning. Adult diagnosis is
often difficult because the outward manifestations most readily evident
to others, especially hyperactivity and impulsivity, often improve
during adolescence and no longer meet diagnostic criteria. The symptoms
of inattention (e.g., easy distractibility, poor organization, being
``spacey,'' avoiding and trouble completing tasks that require sustained
attention, losing things, forgetfulness) are more subtle and may not
reach the level of obvious functional impairment until adulthood, within
an occupational setting or a marriage.

The diagnosis of ADHD in adults, as in childhood, is complicated by the
overlap of symptoms with other disorders. Attention and concentration,
for example, can be impaired in persons who have depression, bipolar
disorder, anxiety, psychosis, post-traumatic stress disorder, or
substance abuse, or in adults who need to perform well in an
overdemanding environment or who are highly stressed or sleep-deprived.
Hyperactivity can be confused with anxiety-related behaviors and the
excessive movements of tic and obsessive-compulsive disorders.
Impulsivity is often prominent in bipolar and substance use disorders.
The accurate diagnosis of adult ADHD is further complicated by
individuals who seek stimulant medications to aid cognitive performance,
especially college students and highly driven working professionals.
Stimulants have long been known to improve sustained attention and
reduce distractibility in healthy individuals who do not have ADHD,
which may prompt success-oriented individuals to feign symptoms in
diagnostic interviews, self-reports, or neuropsychological test
assessments to obtain stimulant medications, and some students feign
illness to receive academic accommodations, such as extended time on
tests, tutoring services, and alternative courses that can improve their
grades.

Claims of exceptional diagnostic performance of these tools, the
differing measures of performance, and the differing performance
characteristics of different versions of a given tool, are controversial
and often confusing to clinicians, patients, and other stakeholders. In
addition, whether the performance of diagnostic tools varies with the
characteristics of the participants with ADHD or comparator sample is
unknown. These diagnostic challenges can complicate the accurate and
reliable diagnosis of adult ADHD even for experienced mental health
clinicians.

Thus, despite established criteria in the Diagnostic and Statistical
Manual of Mental Disorders, Fifth Edition (DSM-5), diagnosing ADHD in
adults remains challenging due to the frequent absence of hyperactivity
and impulsivity symptoms, the subtlety of inattention symptoms, the
inaccuracy of recall in adults for their retrospective assessments of
ADHD symptoms in childhood (required to meet DSM-5 diagnostic criteria),
the common symptom overlap with other mental health conditions, and the
large number of individuals, including healthy college students, who
feign symptoms to obtain stimulant medications. Moreover, the DSM-5
diagnostic criteria, developed primarily for children, may not be
equally suitable for adult diagnosis, and its requirement that symptoms
begin before age 12 has been debated. The absence of a true and
undisputed ``gold-standard'' to verify an ADHD diagnosis, the
variability in performance of diagnostic tools among clinicians and
settings, and the lack of clear practice guidelines further add to
diagnostic complexity.

Furthermore, the diagnosis of ADHD in adults is often made not by mental
health specialists, but by primary care physicians and nurse
practitioners, who may benefit particularly from accurate diagnostic
aids. Further, the dispensing of ADHD medications to adults has
increased steadily over time. The accuracy of diagnosis directly affects
the management and treatment of ADHD and may help prevent medication
misuse, highlighting the need for effective diagnostic tools and
guidelines. The existing standards and guidelines for diagnosing ADHD in
adults are limited, however, and the use of diagnostic tools and
assessments varies widely in practice. No clinical practice guidelines
for the diagnosis of adults with ADHD have thus far been developed in
the United States, though one is in development. Moreover, the
diagnostic accuracy of tools and assessments used in adult ADHD
diagnosis is unclear, and their performance may vary depending on the
characteristics of the ADHD participants and comparator samples.

\subsubsection{1.2 Purpose and Scope}\label{purpose-and-scope}

This systematic review aims to provide a comprehensive and unbiased
assessment of diagnostic tools used to diagnose ADHD in adults to inform
patients, clinicians, and policy makers. Commissioned by the Food and
Drug Administration (FDA), this Agency for Healthcare Research and
Quality (AHRQ) report documents the evidence for the diagnostic
performance of existing tools for ADHD. We explore the effects of
setting and participant characteristics that may influence the
diagnostic performance of available tools. A contextual question is
which tools are frequently being used in current clinical practice.

\begin{center}\rule{0.5\linewidth}{0.5pt}\end{center}

\subsection{2. Methods}\label{methods}

The systematic review followed a protocol that outlines the methods in
detail.

\subsubsection{2.1 Key Questions}\label{key-questions}

\textbf{Key Question 1:} What is the comparative diagnostic accuracy,
unintended consequences, and impact of tools that can be used in the
primary care practice setting or by specialists to diagnose ADHD among
adults?

\textbf{Key Question 1a:} How does the comparative diagnostic accuracy
of these tools vary by clinical setting, including primary care or
specialty clinic, or patient characteristics, including age, sex,
cultural background, and risk factors associated with ADHD?

\subsubsection{2.2 Logic Model}\label{logic-model}

The logic model illustrates the pathway from diagnostic tools to
diagnosis and subsequent treatment decisions for adults with suspected
ADHD.

\subsubsection{2.3 Search Strategy}\label{search-strategy}

We conducted comprehensive searches of nine research databases from
inception through October 2024. We also searched research and guideline
registries, reference-mined existing reviews and practice guidelines,
and consulted with experts to identify relevant studies.

\subsubsection{2.4 Inclusion/Exclusion
Criteria}\label{inclusionexclusion-criteria}

\paragraph{Eligibility Criteria}\label{eligibility-criteria}

\textbf{Inclusion Criteria:} - \textbf{Population:} Adults ≥18 years old
receiving a diagnostic assessment for ADHD, comparator sample or
reference - \textbf{Intervention:} Tests for ADHD: self-report
questionnaire, peer report questionnaire, neuropsychological assessment,
neuroimaging, EEG, diverse biomarkers, clinician tool, combination,
other (e.g., feigning assessment) - \textbf{Comparator:} Tests (as
listed in the intervention) Standardized clinical diagnosis -
\textbf{Outcomes:} Diagnostic accuracy (sensitivity, specificity,
clinical misdiagnosis rate, positive predictive value, negative
predictive value), adverse events, administration time, inter-rater
reliability, costs, diagnostic concordance (primary care vs.~specialist)
- \textbf{Timing:} Diagnosis completed before treatment is initiated -
\textbf{Setting:} Primary or specialty care settings, including
telehealth - \textbf{Study Design:} Diagnostic accuracy studies

\textbf{Exclusion Criteria:} - Studies of populations \textless18 years
old at the time of diagnosis - Diagnosis for nonclinical or not research
purposes - Editorials, nonsystematic reviews, letters, case series, case
reports, pre-post studies - Systematic reviews were not eligible for
inclusion but were retained for reference mining

\paragraph{2.4.1 Screening Process}\label{screening-process}

We used an online database designed for systematic reviews to screen the
literature search output. The team designed detailed citation and full
text screening forms to ensure a transparent, consistent, and
unambiguous approach. All citations were screened by two independent
literature reviewers. Citations found to be potentially relevant by at
least one reviewer were obtained as full text. All citations were also
screened by a DistillerSR software machine learning algorithm trained by
the human reviewers to ensure that no relevant citation was missed.

\subsubsection{2.5 Data Extraction and
Abstraction}\label{data-extraction-and-abstraction}

We captured detailed information about eligible studies. One literature
reviewer extracted data and categorized information where relevant, and
an experienced methodologist checked the data for accuracy and
completeness. The data abstraction documented the targeted population
and characteristics of all included participants (participants with ADHD
and those without). We documented the clinical setting, method of
establishing the reference standard (a clinical ADHD diagnosis), and
diagnostic tool characteristics (format, name of the tool, employed cut
offs, use of a training and validation set).

\subsubsection{2.6 Risk of Bias
Assessment}\label{risk-of-bias-assessment}

The critical appraisal for individual studies applied criteria
consistent with QUADAS 2. QUADAS-2 evaluates four domains:

\begin{itemize}
\tightlist
\item
  \textbf{Patient selection:} Whether the selection of patients could
  have introduced bias
\item
  \textbf{Index test:} Whether the conduct or interpretation of the test
  could have introduced bias
\item
  \textbf{Reference standard:} Whether the reference standard, its
  conduct, or its interpretation may have introduced bias
\item
  \textbf{Flow and timing:} Whether the conduct of the study may have
  introduced bias
\end{itemize}

\subsubsection{2.7 Assessing
Applicability}\label{assessing-applicability}

Results are based on the international literature and applicability
ratings provided assessments regarding the generalizability of samples,
settings, and tool results for U.S. clinical practice.

\subsubsection{2.8 Data Synthesis and
Analysis}\label{data-synthesis-and-analysis}

Data synthesis followed standard systematic review methods. We attempted
meta-analysis where appropriate, considering clinical and methodological
heterogeneity.

\subsubsection{2.9 Grading the Strength of the Body of
Evidence}\label{grading-the-strength-of-the-body-of-evidence}

We assessed the strength of evidence for key outcomes using established
criteria considering study limitations, directness, consistency,
precision, and reporting bias.

\begin{center}\rule{0.5\linewidth}{0.5pt}\end{center}

\subsection{3. Results}\label{results}

\subsubsection{3.1 Results of Literature
Search}\label{results-of-literature-search}

We identified 117 studies evaluating the diagnostic performance of tools
for ADHD diagnosis in adults.

\subsubsection{3.2 Results of Key Question 1: Comparative Diagnostic
Accuracy}\label{results-of-key-question-1-comparative-diagnostic-accuracy}

\paragraph{3.2.1 Combination}\label{combination}

Studies evaluating combinations of diagnostic tools showed promising
results. Combinations of input indicated a fair rate of clinical false
positive rates, good sensitivity, and acceptable specificity.

\paragraph{3.2.2 Self-Report
Questionnaires}\label{self-report-questionnaires}

Self-report questionnaires were the most commonly evaluated diagnostic
tools. They showed good sensitivity and specificity, though often not
both in the same study. Administration time was typically short (10-20
minutes), but agreement with other raters was limited.

\paragraph{3.2.3 Peer Report
Questionnaires}\label{peer-report-questionnaires}

Peer report questionnaires showed limited specificity. Inter-rater
reliability varied substantially across studies.

\paragraph{3.2.4 Neuropsychological
Assessment}\label{neuropsychological-assessment}

Neuropsychological tests reported substantial false positive rates in
clinical samples, with acceptable sensitivity and specificity overall.
Administration times were typically around 20 minutes.

\paragraph{3.2.5 Neuroimaging}\label{neuroimaging}

Five studies evaluated neuroimaging techniques including SPECT, MRI, and
functional MRI. Clinical misdiagnosis rates varied from 3\% to 24\%.
Reported sensitivity ranged from 54\% to 100\%, with most studies
reporting acceptable specificity.

\paragraph{3.2.6 EEG}\label{eeg}

Twelve studies evaluated EEG data for distinguishing ADHD from other
conditions. Only one study reported on misdiagnosis in a clinical sample
(3.8\% false positive rate). Reported sensitivity ranged from 67\% to
100\%, with generally good specificity.

\paragraph{3.2.7 Biomarker}\label{biomarker}

Five studies evaluated various biomarkers including genetic markers, eye
tracking, blood oxidative status, physiological data from wearables, and
motor function assessments. Results varied widely with no consistent
patterns.

\paragraph{3.2.8 Clinician Tool}\label{clinician-tool}

Three studies reported on clinician interviews or questionnaires. These
tools generally showed fair sensitivity but variable specificity.

\paragraph{3.2.9 Key Question 1a: Variation by Setting and Patient
Characteristics}\label{key-question-1a-variation-by-setting-and-patient-characteristics}

Evidence was insufficient to determine how diagnostic accuracy varies by
clinical setting or patient characteristics due to limited comparative
data.

\paragraph{3.2.10 Key Question 1 Summary of
Findings}\label{key-question-1-summary-of-findings}

The evidence suggests that combinations of diagnostic tools may offer
the best overall performance. Self-report questionnaires are widely
studied but show variable performance. Neuropsychological tests have
acceptable performance but substantial false positive rates in clinical
samples. The evidence for newer modalities like neuroimaging and EEG is
limited but shows promise.

\begin{center}\rule{0.5\linewidth}{0.5pt}\end{center}

\subsection{4. Discussion}\label{discussion}

\subsubsection{4.1 Comparative Diagnostic Performance of
Tools}\label{comparative-diagnostic-performance-of-tools}

\paragraph{4.1.1 Measures Reported for Diagnostic
Performance}\label{measures-reported-for-diagnostic-performance}

Studies used various measures of diagnostic performance, making direct
comparisons challenging. The most commonly reported measures were
sensitivity and specificity, though clinical misdiagnosis rates (false
positive rates in clinical samples) may be more clinically relevant.

\paragraph{4.1.2 The Importance of the Comparator
Sample}\label{the-importance-of-the-comparator-sample}

Diagnostic performance varied substantially depending on whether the
comparator group consisted of neurotypical adults or adults with other
clinical conditions. Tools that perform well in distinguishing ADHD from
neurotypical development may perform poorly when distinguishing ADHD
from other psychiatric conditions.

\paragraph{4.1.3 Rating Scales}\label{rating-scales}

Self-report and peer-report rating scales are the most extensively
studied diagnostic tools. While they offer practical advantages
including brief administration time and low cost, their performance
varies substantially across studies and settings.

\paragraph{4.1.4 Neuropsychological
Tests}\label{neuropsychological-tests}

Neuropsychological tests provide objective measures of cognitive
function but show substantial false positive rates in clinical samples,
limiting their utility as standalone diagnostic tools.

\paragraph{4.1.5 Other Diagnostic Tools}\label{other-diagnostic-tools}

Emerging diagnostic modalities including neuroimaging, EEG, and various
biomarkers show promise but require further validation in diverse
clinical populations.

\subsubsection{4.2 Direct Comparisons of Diagnostic
Performance}\label{direct-comparisons-of-diagnostic-performance}

Few studies directly compared different diagnostic tools within the same
population. The limited comparative evidence suggests that combinations
of tools outperform individual tools, and self-report questionnaires may
have lower false positive rates than neuropsychological tests in
clinical samples.

\subsubsection{4.3 Implications}\label{implications}

The findings have several important implications for clinical practice:

\begin{enumerate}
\def\labelenumi{\arabic{enumi}.}
\tightlist
\item
  No single diagnostic tool demonstrates sufficient accuracy to serve as
  a standalone diagnostic method for adult ADHD
\item
  Combinations of diagnostic tools appear to offer superior performance
  compared to individual tools
\item
  The choice of diagnostic tools should consider the clinical context,
  particularly whether the goal is to distinguish ADHD from neurotypical
  development or from other psychiatric conditions
\item
  Standardization of diagnostic approaches and validation in diverse
  clinical populations is needed
\end{enumerate}

\subsubsection{4.4 Strengths, Limitations, and
Applicability}\label{strengths-limitations-and-applicability}

\textbf{Strengths:} - Comprehensive search strategy - Systematic
assessment of risk of bias and applicability - Inclusion of diverse
diagnostic modalities

\textbf{Limitations:} - Heterogeneity in study populations, comparators,
and outcome measures - Limited head-to-head comparisons of diagnostic
tools - Variation in reference standards used across studies

\textbf{Applicability:} - Results are broadly applicable to U.S.
clinical practice - However, many tools evaluated in research settings
may not be readily available in routine clinical practice

\subsubsection{4.5 Next Steps}\label{next-steps}

Future research priorities include: 1. Direct comparison studies of
diagnostic tools in the same populations 2. Validation of diagnostic
tools in primary care settings 3. Development and validation of
diagnostic algorithms that integrate multiple tools 4. Studies examining
the impact of diagnostic tools on clinical outcomes and treatment
decisions 5. Evaluation of diagnostic tools in diverse populations
including different age groups, cultural backgrounds, and comorbidity
profiles

\begin{center}\rule{0.5\linewidth}{0.5pt}\end{center}

\subsection{References}\label{references}

{[}References section would include the full bibliography - not
extracted in this version as requested to focus on main text{]}

\begin{center}\rule{0.5\linewidth}{0.5pt}\end{center}

\subsection{Abbreviations and
Acronyms}\label{abbreviations-and-acronyms}

\begin{itemize}
\tightlist
\item
  ADHD: Attention-Deficit/Hyperactivity Disorder
\item
  AHRQ: Agency for Healthcare Research and Quality
\item
  AUC: Area Under the Curve
\item
  DSM-5: Diagnostic and Statistical Manual of Mental Disorders, Fifth
  Edition
\item
  EEG: Electroencephalogram
\item
  EPC: Evidence-based Practice Center
\item
  FDA: Food and Drug Administration
\item
  MRI: Magnetic Resonance Imaging
\item
  SPECT: Single-Photon Emission Computed Tomography
\item
  SoE: Strength of Evidence
\end{itemize}



\section{Supplemental Figures}\label{supplemental-figures}

The section numbering will be changed to ``A.1.1'' in the appendix. The
second section in the appendix will be ``B''. On the other hand, the
figure numbering will be reset to ``A.1'', ``A.2'' so that it is clear
that these figures are part of the appendix. The ``A'' stands for the
``Appendix'', not the section numbering.

\newpage{}




\end{document}
